

\section{Tall}

De første tallene som ble brukt av menneskeheten var de \emph{naturlige tallene}. 
Disse brukes for å telle ting, og består av tallene 
$$1, 2, 3, 4, 5, \ldots$$
og så videre.
Vi legger også til tallet $0$, som brukes for å symbolisere at vi ikke har tellt noe, eller at vi har abosolutt ingenting av et eller annet. 

Historisk sett kom man veldig langt med å kun bruke disse tallene, men etterhvert i menneskets utvikling begynte man å lage mer rigorøse pengesystemer. 
Disse inkorporerte konseptet med gjeld, og etterhvert var det beleielig å introdusere nye tall som kunne representere slik gjeld. 
Disse tallene er de negative tallene, 
$$-1, -2, -3, -4, -5, \ldots.$$
Sammen med de naturlige tallene danner disse de såkalte \emph{heltallene}, 
$$\ldots, -3, -2, -1, 0, 1, 2, 3, \ldots . $$

Litt videre i historien dukket det opp behov for mer nøyanse i tallsystemene som ble brukt. 
Ofte var det nødvendig å skrive ned tall som for eksempel betegnet bare en liten del av noe større, eller for å betegne et forhold mellom to størrelser. 
Slike konsepter beskrives med de såkalte \emph{rasjonale tallene}. 
Ordet ``rasjonalt'' kommer i denne sammenhengen fra engelsk, der tallene heter \emph{rational numbers}. 
Ordet ``ratio'' betyr ``forhold'', som symbolisere at disse tallene beskriver forhold mellom ting. 
Det samme ordet brukes ofte til å beskrive en brøk, så på norsk kunne vi egentlig kalt disse tallene for \emph{forholdstallene} eller \emph{brøktallene}.





\subsection{Oppsummering}
\begin{itemize}
    \item $\N = \{0,1,2,3,4,\ldots \}$
    \item $\Z = \{\ldots, -2, -1, 0, 1, 2, \ldots\}$ 
    \item $\Q = \{\frac{a}{b} \mid a, b \in \Z\}$
\end{itemize}